\documentclass[12pt,a4paper]{article}

\usepackage[utf8]{inputenc}
\usepackage[french]{babel}
\usepackage{amsmath}
\usepackage{amsfonts}
\usepackage{amssymb}
\usepackage{graphicx}
\usepackage{geometry}
\usepackage{listings}
\usepackage{xcolor}
\usepackage{float}
\usepackage{hyperref}
\usepackage{tcolorbox}
\usepackage{fontawesome}
\usepackage{tikz}
\usepackage{enumitem}
\usepackage{fancyhdr}
\usepackage{titlesec}
\usepackage{multicol}
\usepackage{booktabs}
\usepackage{array}

\geometry{margin=2cm, top=3cm, bottom=3cm}

% Couleurs modernes
\definecolor{primaryblue}{RGB}{41, 128, 185}
\definecolor{secondaryblue}{RGB}{52, 152, 219}
\definecolor{accentgreen}{RGB}{39, 174, 96}
\definecolor{warningorange}{RGB}{230, 126, 34}
\definecolor{darkgray}{RGB}{44, 62, 80}
\definecolor{lightgray}{RGB}{236, 240, 241}
\definecolor{codebg}{RGB}{248, 249, 250}
\definecolor{codeborder}{RGB}{220, 221, 222}

% Configuration des listes
\setlist[itemize]{leftmargin=*, itemsep=0.5em}
\setlist[enumerate]{leftmargin=*, itemsep=0.5em}

% Configuration des liens
\hypersetup{
    colorlinks=true,
    linkcolor=primaryblue,
    urlcolor=secondaryblue,
    citecolor=accentgreen
}

% En-têtes et pieds de page
\pagestyle{fancy}
\fancyhf{}
\fancyhead[L]{\textcolor{primaryblue}{\textbf{Projet Automatisation E-commerce}}}
\fancyhead[R]{\textcolor{darkgray}{Ahmed \& Moussa}}
\fancyfoot[C]{\textcolor{darkgray}{\thepage}}
\renewcommand{\headrulewidth}{2pt}
\renewcommand{\headrule}{\hbox to\headwidth{\color{primaryblue}\leaders\hrule height \headrulewidth\hfill}}

% Configuration des titres
\titleformat{\section}
{\color{primaryblue}\Large\bfseries}
{\color{primaryblue}\thesection}{1em}{}
[\color{primaryblue}\titlerule]

\titleformat{\subsection}
{\color{secondaryblue}\large\bfseries}
{\color{secondaryblue}\thesubsection}{1em}{}

\titleformat{\subsubsection}
{\color{darkgray}\normalsize\bfseries}
{\color{darkgray}\thesubsubsection}{1em}{}

% Configuration des blocs de code TypeScript
\lstset{
    backgroundcolor=\color{codebg},
    commentstyle=\color{accentgreen},
    keywordstyle=\color{primaryblue}\bfseries,
    numberstyle=\tiny\color{darkgray},
    stringstyle=\color{warningorange},
    basicstyle=\ttfamily\small,
    breakatwhitespace=false,
    breaklines=true,
    captionpos=b,
    keepspaces=true,
    numbers=left,
    numbersep=5pt,
    showspaces=false,
    showstringspaces=false,
    showtabs=false,
    tabsize=2,
    frame=single,
    rulecolor=\color{codeborder},
    language=TypeScript
}

% Style pour HTML
\lstdefinestyle{html}{
    backgroundcolor=\color{codebg},
    commentstyle=\color{accentgreen},
    keywordstyle=\color{primaryblue}\bfseries,
    numberstyle=\tiny\color{darkgray},
    stringstyle=\color{warningorange},
    basicstyle=\ttfamily\small,
    breakatwhitespace=false,
    breaklines=true,
    captionpos=b,
    keepspaces=true,
    numbers=left,
    numbersep=5pt,
    showspaces=false,
    showstringspaces=false,
    showtabs=false,
    tabsize=2,
    frame=single,
    rulecolor=\color{codeborder},
    language=HTML
}

% Style pour bash
\lstdefinestyle{bash}{
    backgroundcolor=\color{codebg},
    commentstyle=\color{accentgreen},
    keywordstyle=\color{primaryblue}\bfseries,
    numberstyle=\tiny\color{darkgray},
    stringstyle=\color{warningorange},
    basicstyle=\ttfamily\small,
    breakatwhitespace=false,
    breaklines=true,
    captionpos=b,
    keepspaces=true,
    numbers=left,
    numbersep=5pt,
    showspaces=false,
    showstringspaces=false,
    showtabs=false,
    tabsize=2,
    frame=single,
    rulecolor=\color{codeborder},
    language=bash
}

% Style pour les blocs d'information
\newtcolorbox{infobox}{
    colback=lightgray,
    colframe=primaryblue,
    boxrule=2pt,
    arc=3pt,
    left=10pt,
    right=10pt,
    top=10pt,
    bottom=10pt
}

\newtcolorbox{warningbox}{
    colback=warningorange!10,
    colframe=warningorange,
    boxrule=2pt,
    arc=3pt,
    left=10pt,
    right=10pt,
    top=10pt,
    bottom=10pt
}

\newtcolorbox{successbox}{
    colback=accentgreen!10,
    colframe=accentgreen,
    boxrule=2pt,
    arc=3pt,
    left=10pt,
    right=10pt,
    top=10pt,
    bottom=10pt
}

\begin{document}

% Page de titre moderne (gabarit)
\begin{titlepage}
\centering
\vspace*{2cm}
% Logo/Titre principal
{\Huge\bfseries\color{primaryblue}Projet 2 : Automatisation de la recherche e-commerce}\\[0.5cm]
{\Large\color{secondaryblue}Robot Framework \& SeleniumLibrary pour la boutique Lumina}\\[2cm]

% Informations étudiant
\begin{tcolorbox}[colback=lightgray, colframe=primaryblue, boxrule=2pt, arc=3pt, width=0.6\textwidth]
\centering
\vspace{0.5cm}
{\large\bfseries\color{darkgray}Établissement :}\\[0.3cm]
{\LARGE\color{primaryblue}ENET'COM}\\[0.5cm]
{\large\bfseries\color{darkgray}Étudiants :}\\[0.3cm]
{\LARGE\color{primaryblue}Ahmed AbdDAyem AHMEDBOUHA}\\[0.2cm]
{\Large\color{secondaryblue}Moussa Mahmoud BA}\\[0.5cm]
{\large\color{darkgray}Classe : 3 GT TST}\\[0.3cm]
{\large\color{darkgray}Enseignante : Mme INES Jemal}\\[0.3cm]
{\large\color{darkgray}Année universitaire : 2024/2025}
\vspace{0.5cm}
\end{tcolorbox}

\vfill
\end{titlepage}

\newpage
\tableofcontents
\newpage

\section{Introduction}
\begin{infobox}
Ce document couvre le \textbf{Projet 2} donné par l'enseignante : automatisation de la fonction de recherche sur un site e-commerce (boutique Lumina). Les tests sont écrits avec Robot Framework et SeleniumLibrary pour valider l'expérience utilisateur côté front.
\end{infobox}

\section{Vue d'ensemble du projet}
\subsection{Objectif}
Automatiser la validation de la recherche produit (résultats existants, catégories, aucun résultat, correspondance partielle) pour garantir fiabilité et régression rapide.

\subsection{Périmètre}
\begin{itemize}
    \item Site vitrine statique (HTML/CSS/JS) avec logique de recherche côté client.
    \item Tests end-to-end via navigateur piloté par SeleniumLibrary.
    \item Génération de rapports Robot Framework dans le dossier \texttt{results/}.
\end{itemize}

\section{Stack technique}
\begin{itemize}
    \item \textbf{Frontend} : HTML5, CSS3 (glassmorphism), JavaScript vanilla.
    \item \textbf{Automation} : Robot Framework, SeleniumLibrary, Python.
    \item \textbf{Outils} : Navigateur contrôlé par WebDriver.
\end{itemize}

\section{Structure du dépôt}
\begin{lstlisting}[style=bash, caption=Arborescence]
projetTest/
├── ecommerce_project/
│   ├── website/          # Application sous test
│   │   ├── index.html
│   │   ├── style.css
│   │   └── script.js
│   └── tests/            # Suites Robot Framework
│       ├── search_tests.robot
│       └── resources.robot
└── results/              # Rapports d'exécution
\end{lstlisting}

\section{Mise en route}
\subsection{Prérequis}
\begin{lstlisting}[style=bash, caption=Installation des dépendances]
pip install robotframework robotframework-seleniumlibrary
\end{lstlisting}

\subsection{Lancer le site local}
\begin{lstlisting}[style=bash, caption=Serveur local]
cd ecommerce_project/website
python3 -m http.server 8000
# Accès: http://localhost:8000
\end{lstlisting}

\subsection{Exécuter les tests}
\begin{lstlisting}[style=bash, caption=Lancer la suite Robot]
robot -d results ecommerce_project/tests/search_tests.robot
\end{lstlisting}
Les rapports HTML (log/report) sont générés dans \texttt{results/}.

\section{Scénarios testés}
\begin{itemize}
    \item \textcolor{primaryblue}{Produit existant} : vérifie l'affichage correct des résultats.
    \item \textcolor{accentgreen}{Recherche par catégorie} : filtrage (ex. ``Audio'').
    \item \textcolor{warningorange}{Aucun résultat} : état vide et message utilisateur.
    \item \textcolor{secondaryblue}{Correspondance partielle} : tolérance aux mots partiels.
\end{itemize}

\section{Principaux mots-clés Robot (extraits)}
\begin{itemize}
    \item Ouverture du navigateur et navigation vers \texttt{http://localhost:8000}.
    \item Saisie d'un terme de recherche et déclenchement de la recherche.
    \item Vérification du nombre et du contenu des cartes produits affichées.
    \item Validation du message d'état quand aucun produit n'est trouvé.
\end{itemize}

\section{Captures / Reporting}
\begin{itemize}
    \item Les exécutions produisent \texttt{report.html} et \texttt{log.html} dans \texttt{results/}.
    \item Les captures d'écran peuvent être ajoutées dans \texttt{results/} via SeleniumLibrary si nécessaire.
\end{itemize}

\section{Captures d'écran}
\begin{figure}[H]
    \centering
    \includegraphics[width=0.9\textwidth]{captures/homepage.png}
    \caption{\textcolor{primaryblue}{\textbf{Figure 1 :}} Accueil de la boutique (recherche par défaut)}
\end{figure}

\begin{figure}[H]
    \centering
    \includegraphics[width=0.9\textwidth]{captures/search-existing.png}
    \caption{\textcolor{accentgreen}{\textbf{Figure 2 :}} Résultats pour un produit existant}
\end{figure}

\begin{figure}[H]
    \centering
    \includegraphics[width=0.9\textwidth]{captures/search-empty.png}
    \caption{\textcolor{warningorange}{\textbf{Figure 3 :}} État vide quand aucun produit ne correspond}
\end{figure}

\begin{warningbox}
Remplacez les chemins d'image si besoin (\texttt{captures/}) et ajustez les légendes selon vos propres captures.
\end{warningbox}

\section{Conclusion}
\begin{successbox}
L'automatisation garantit la fiabilité de la recherche e-commerce et accélère les cycles de régression. Les prochains pas peuvent inclure des scénarios multi-navigateurs et l'intégration continue.
\end{successbox}

\end{document}

